\chapter{Conclusion}

Despite these limitations and constraints, our team demonstrated adaptability and commitment to overcoming these obstacles. We implemented time management strategies, intensified our research, shared knowledge and worked autonomously when necessary. We also made strategic decisions to prioritize tasks and optimize our robot access time.\\
\\
It's also important to emphasize that our project differs from conventional programming projects. Our approach is based more on research and exploration of the psychological and behavioral aspects of human-robot interaction. This less conventional approach proved to be a wise choice, as it not only enabled us to acquire new knowledge and skills, but also piqued our interest in broader areas of computer science and robotics.\\
\\
It's worth pointing out that, despite the research orientation of our project, we still discovered many new technologies. We discovered what it was like to work with a real robot. We developed programming skills to set up the virtual reality experience and adapt the QTrobot's behavior. We also honed our modeling skills with the photogrammetry method. These complementary technical skills have enabled us to concretize our innovative research approach, combining theoretical and practical aspects for a better understanding of human-robot interaction.\\
\\
This multi-disciplinary approach has broadened our horizons and encouraged us to pursue our involvement in broader areas of computer science and robotics, where consideration of human factors is essential.\\
\\
In conclusion, our experience has demonstrated the importance of taking human and emotional aspects into account in the development of human-robot interactions. The results obtained and the knowledge acquired in the course of this project pave the way for new research and development perspectives in the field of human-robot interaction. We are proud to have taken part in this unique experiment, and are convinced that human-robot interaction will be at the heart of tomorrow's debates.