\chapter{Introduction}
\subsection{Genesis of the project}
The project was born of a common idea: to create an authentic experience involving QtRobot and virtual reality (VR). After extensive research into the principle of embodiment and robots, we selected a theme for the experience. Initially, we had envisaged a play between the participant in the experiment and the robot. However, the complexity of the project lay in its freedom. As a result, we rethought the dialogue of the play, in order to obtain something closer to a natural discussion, so we abandoned the idea of a play. We then tested the ChatGPT discussions, and found that it was not easy to analyze how people felt during these exchanges, as they varied considerably depending on the topics discussed. To achieve our main objective, which was to carry out a genuine experiment with an ethical framework, a diversity of interesting data and different groups of people, we took the decision to restrict the field of discussion to one game to ensure a similar experience for all. We opted for the "no yes, no no" game, which we felt was more practical for gathering information on emotional reactions.\\

\subsection{Specifications}
Although we were rather free in the conception of our project, we decided to draw up specifications.\\

\subsubsection{Expected features}
\begin{enumerate}
\item Functional program on the QTrobot: develop an operational program on the QTrobot that will enable fluid, immersive interaction to play the "no yes, no no" game directly with the robot.

\item Functional program in virtual reality (VR): development of a functional program in VR to also play the "no yes, no no" game. This will enable users to enjoy a similar experience, but in a virtual environment, by interacting with an avatar or virtual representative of the robot.

\item Creating a real experience: creating a truly immersive and engaging experience for users. This involves setting up an ethical form to collect data from participants, ensuring their informed consent and the confidentiality of their information. In addition, the aim is to collect varied and interesting data sets to better understand participants' emotional reactions and behaviors when interacting with the robot or in virtual reality.
\end{enumerate}
\subsubsection{Constraints}
From the outset, we identified a number of specific constraints that we wanted to incorporate into the project.
\begin{enumerate}
    \item Robot language learning (ROS) was a major constraint in this project. QTrobot is an interactive robot with natural language functionalities. In order to program interactions between the robot and users, a thorough knowledge of the fundamentals of robot language was required. This included understanding the concepts of speech recognition, natural language processing and speech generation.
    
    \item Another constraint was learning to use virtual reality (VR) tools. In this project, we used virtual reality to simulate the interactive environments in which QTrobot evolves. This involved acquiring in-depth knowledge of VR technologies, development environments and man-machine interaction techniques specific to VR.

    \item Learning 3D modeling was another major constraint. We created detailed 3D models of QTrobot for visualization and simulation.
    
    \item Finally, the use of an agile method was an essential constraint in managing the project. The agile method enabled us to work iteratively and collaboratively, constantly adapting our approach to the challenges encountered. This enabled us to remain flexible and responsive to change throughout the project development process.
\end{enumerate}